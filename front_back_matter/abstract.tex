% Abstract

\pdfbookmark[1]{Abstract}{Abstract} % Bookmark name visible in a PDF viewer

\begingroup
\let\clearpage\relax
\let\cleardoublepage\relax
\let\cleardoublepage\relax

\chapter*{Abstract} % Abstract name

In the past, making data tangible was a complicated, manual process. Digital 3D
representations of complex data have been around for quite a while, but they
were always confined to the digital world. Mostly because it was impractical to
convert a digital model to a physical representation.

With the advent of cheap, affordable 3D printers, this changed. It is now easy
to convert a purely digital model to a tangible, physical object. The missing
piece in the process of making data tangible is the conversion of data to a
digital 3D model.

This thesis wants to solve that problem by providing an easy to use software
library with ``batteries included'' that can convert arbitrary numeric data to
3D models. The library -- named \tangible{} -- is written in Python and provides
a set of predefined but customizable shapes, a few tools to preprocess data and
a backend implementation for OpenSCAD, an open source programmatic CAD software.

\tangible{} is implemented as a cross-compiler with a simple abstract syntax
tree (AST), a set of predefined shapes that build on top of the AST and an
interface that allows the creation of different code generation backends.

The library is ready to use, well tested and thoroughly documented. It has been
released under an open source license and is available online at
\url{https://github.com/dbrgn/tangible}.

\endgroup			

\paragraph{Keywords:}\mbox{}\\
\textit{\myKeywords}

\vfill
