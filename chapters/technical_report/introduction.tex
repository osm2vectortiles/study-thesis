\chapter{Introduction}

%-------------------------------------------------------

Creating a custom styled OSM map is one of the most common use cases
among cartographers yet it is very difficult to do so. With the new
emerging technology of vector tiles it is possible to allow anyone to
create their custom OSM maps without setting up a database and
managing complex infrastructure.

%------------------------------------------------------

\section{Vision}\label{vision}

Michal Migurski published on March 15, 2013 a blog
post\footnote{\url{http://mike.teczno.com/notes/postgreslessness-mapnik-vectiles.html}},
in which he described his first attempts to use vector tiles as a source
for the Mapnik tile renderer and his vision for vector tiles.

Vector tiles only contain geometries and metadata. The visual style can
be applied to a tile when it is requested. Since vector tiles do not
contain any information about the style, they are smaller than raster
tiles. This enables high resolution maps, fast map loads and efficient
caching.

Our mission is to bring the power of vector tiles to anyone and provide
the data free and as Open Source project.

%------------------------------------------------------

\section{Goals}\label{targets}

The main goal of this thesis is to allow anyone to create their custom
OSM map without managing complex infrastructure. In order to complete this goal
several deliverables were defined in the project proposal.

\begin{itemize}
\item
  Deliver a Docker Container to create vector tiles from \osm{} data
\item
  Provide vector tiles for Switzerland
\item
  Provide a Docker container to serve the vector tiles together with
  custom styles as raster tiles
\item
  Optional: Vector tiles for the whole world
\end{itemize}

%--------------------------------------------------------
\newpage
\section{Project Procedure}

The project was split into the following steps:

\begin{itemize}
\item
  Inception: Kickoff meeting with Petr Pridal, Stefan Keller and
  definition of project proposal
\item
  Prototyping: Get to know the entire Mapbox software stack and create
  a working prototype of every deliverable.
\item
  Evaluation and Requirements: Evaluation of different parts of the
  stack like import tools and vector tile server. The evaluation of
  this technology as well as additional technologies are part of the
  section technology evaluation.
\item
  Implementation: The creation of the map was an iterative process. We
  started at the lowest zoom level 14 and implemented level by level up
  to zoom level 0. On the way we identified many different problems
  which are further described in the section implementation of the
  project documentation.
\item
  Optimization: After the first alpha version
  additional data sources (Natural Earth) were added to further improve the quality
  of the upper zoom levels. During the implementation phase the project structure
  has been refactored and continuous integration was implemented.
\item
  Rendering: At the end of the project there was a long period for the
  actual rendering of the vector tiles.
\end{itemize}
\newpage