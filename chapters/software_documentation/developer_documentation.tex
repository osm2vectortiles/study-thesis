\chapter{Developer Documentation}

\section*{Create your own vector tiles}

Docker is used extensively for development and deployment. The easiest
way to get started is using
Docker Compose\footnote{\url{https://www.docker.com/docker-compose}}.
\\\\
1. Step: Clone the osm2vectortiles project.

\begin{bashcode}
git clone https://github.com/osm2vectortiles/osm2vectortiles.git
\end{bashcode}

2. Step: Start up your PostGIS container with the data container attached.

\begin{bashcode}
docker-compose up -d postgis
\end{bashcode}

3. Step: Download a PBF and put it into the local \texttt{import} directory.

\begin{bashcode}
wget https://s3.amazonaws.com/metro-extracts.mapzen.com/zurich_switzerland.osm.pbf
\end{bashcode}

4. Step: Now you need to import the PBF files into PostGIS.

\begin{bashcode}
docker-compose up import-osm
\end{bashcode}

5. Step: Now you need to import several external data sources.

Import water polygons from OpenStreetMapData.

\begin{bashcode}
docker-compose up import-water
\end{bashcode}

6. Step: Import Natural Earth data for lower zoom levels.

\begin{bashcode}
docker-compose up import-natural-earth
\end{bashcode}

7. Step: Import custom country, sea and state labels.

\begin{bashcode}
docker-compose up import-labels
\end{bashcode}

8. Step: Now import custom SQL functions.

\begin{bashcode}
docker-compose up import-sql
\end{bashcode}

9. Step: Update the scaleranks of OSM places.

\begin{bashcode}
docker-compose up update-scaleranks
\end{bashcode}

10. Step: Export the data as MBTiles file to the \texttt{export} directory.

\begin{bashcode}
docker-compose up export
\end{bashcode}

11. Step: Serve the tiles as raster tiles from \texttt{export} directory.

\begin{bashcode}
docker-compose up serve
\end{bashcode}



\section*{Create own extract}\label{create-own-extract}

If you need an extract which is not included on the
downloads page\footnote{\url{http://osm2vectortiles.org/data/download.html}},
you can download the planet file and make your own extract.

\subsection*{Preparation}

\begin{enumerate}
\item
  Download\footnote{\url{http://osm2vectortiles.org/data/download.html}} the
  planet file.
\item
  Get the bounding box of your desired extract.\footnote{\url{http://tools.geofabrik.de/calc/\#type=geofabrik_standard\&bbox=5.538062,47.236312,15.371071,54.954937\&tab=1\&proj=EPSG:4326\&places=2}}
\item
  Install tilelive utility.

\begin{bashcode}
npm install -g tilelive
\end{bashcode}
\end{enumerate}

\subsection*{Create Extract}\label{create-extract}

To create an extract the \texttt{tilelive-copy} utility is used. It takes a
bounding box and a MBTiles file as input and ouputs the extract.

Replace the bounding box in the following command with your bounding
box.

\begin{bashcode}
tilelive-copy --minzoom=0 --maxzoom=14 --bounds="60.403889,29.288333,74.989862,38.5899217" 
world.mbtiles switzerland.mbtiles
\end{bashcode}
\newpage

\section*{Layer Reference}\label{layer-reference}

This is a guide to the data inside the OSM Vector Tiles to help you with
styling.

Available layers:

\begin{itemize}
\item
  landuse
\item
  waterway
\item
  water
\item
  aeroway
\item
  barrier\_line
\item
  building
\item
  landuse\_overlay
\item
  tunnel
\item
  road
\item
  bridge
\item
  admin
\item
  country\_label
\item
  marine\_label
\item
  place\_label
\item
  water\_label
\item
  poi\_label
\item
  road\_label
\item
  waterway\_label
\item
  housenum\_label
\end{itemize}

\subsection*{landuse}\label{landuse}

This layer includes polygons representing both land-use and land-cover.

It's common for many different types of landuse/landcover to be
overlapping, so the polygons in this layer are ordered by the area of
their geometries to ensure smaller objects will not be obscured by
larger ones. Pay attention to use of transparency when styling - the
overlapping shapes can cause muddied or unexpected colors.

\subsection*{water}\label{water}

This is a simple polygon layer with no differentiating types or classes.
Water bodies are filtered and simplified according to scale - only
oceans and very large lakes are shown at the lowest zoom levels, while
smaller and smaller lakes and ponds appear as you zoom in.

\subsection*{waterway}\label{waterway}

The waterway layer contains rivers, streams, canals, etc represented as
lines.

\subsubsection*{Type}

The \texttt{type} column can contain one of the following types:

\begin{itemize}
\item
  \texttt{ditch}
\item
  \texttt{stream}
\item
  \texttt{stream\_intermittent}
\item
  \texttt{river}
\item
  \texttt{canal}
\item
  \texttt{drain}
\item
  \texttt{ditch}
\end{itemize}

\subsection*{aeroway}\label{aeroway}

The aeroway layer contains the types runway, taxiway, apron and helipad.

\subsection*{barrier\_line}\label{barrierux5fline}

The barrier\_line layer contains the classes fence, cliff, gate.

\subsection*{building}\label{building}

This layer contains buildings. Buildings are shown starting zoom level
13.

\subsection*{landuse\_overlay}\label{landuseux5foverlay}

This layer is for landuse polygons that should be drawn above the
\#water layer.

\subsection*{tunnel, road, bridge}\label{tunnel-road-bridge}

The layers tunnel and bridge are based of the layer road.

\subsubsection*{Class}

The main field used for styling the tunnel, road, and bridge layers is the \texttt{class} field.

\begin{table}[H]
\label{my-label}
\begin{tabular}{p{3cm} | p{8cm}}
Class           & Aggregated Types                                                                                                              \\ \hline
motorway        & motorway                                                                                                                      \\
motorway\_link  & motorway\_link                                                                                                                \\
driveway        & driveway                                                                                                                      \\
main            & primary, primary\_link, trunk, trunk\_link, secondary, secondary\_link, tertiary, tertiary\_link                              \\
street          & residential, unclassified, living\_street, road, raceway                                                                      \\
street\_limited & pedestrian, construction, private                                                                                             \\
service         & service, track, alley. spur, siding, crossover                                                                                \\
path            & path, cycleway, ski, steps, bridleway, footway                                                                                \\
major\_rail     & rail, monorail, narrow\_gauge, subway                                                                                         \\
aerialway       & chair\_lift, mixed\_lift, drag\_lift, platter, t-bar, magic\_carpet, gondola, cable\_car, rope\_tow, zip\_line, j-bar, canopy \\
golf            & hole                                                                                                                          \\
                &                                                                                                                              
\end{tabular}
\end{table}

\subsection*{admin}

This layer contains the administrative boundary lines. These are based
on Natural Earth data on lower zoom levels (0-6) and OSM data (7-14) on
upper zoom levels.

\subsubsection*{Administrative Levels}

The administrative levels are in the \texttt{admin} field. 

\begin{table}[H]
\label{my-label}
\begin{tabular}{l | l}
Value & Aggregated Types  \\ \hline
2     & countries         \\
4     & states, provinces
\end{tabular}
\end{table}

The \texttt{disputed} field should be used to apply a dashed or otherwise
distinct style to disputed boundaries.

\subsubsection*{Maritime Boundaries}

The \texttt{maritime} field can be used as a filter to downplay or hide maritime
boundaries, which are often not shown on maps.

\subsubsection*{Localrank}

The \texttt{country\_label} layer contains the labels of all countries with
translated names.

\subsubsection*{Localrank}

The \texttt{scalerank} field is used to hide or show the label based on the
importance, size and available room.

\subsection*{marine\_label}\label{marine_label}

The marine\_label layer contains labels for marine features such as
oceans, seas, large lakes and bays.
The \texttt{labelrank} is used to hide or show the label based on the importance,
size and available room.

\subsection*{state\_label}\label{state_label}

The layer state\_label contains labels for large provinces in large
countries such as China, USA, Russia, Australia and UK.

\subsection*{place\_label}\label{place_label}

The layer place\_label contains labels for cities.

\subsubsection*{Scalerank}

The scalerank is used to hide or show the label based on the importance,
size and available room.

\subsubsection*{Localrank}

The \texttt{localrank} field can be used to adjust the label density by showing
fewer labels.

\subsection*{water\_label}

The layer water\_label contains labels for bodies of water such as
lakes.

\subsection*{road\_label}

The layer road\_label uses the same classification as the layer road.

\subsection*{waterway\_label}

The layer waterway\_label contains labels for rivers.

\subsection*{housenum\_label}

This layer contains points used to label the street number parts of
specific addresses. Both housenumber polygons and points were mapped to
a single layer.
The \texttt{house\_num} field countains house and building numbers.

\subsection*{poi\_label}

The poi\_label layer is used to place icons and labels for various point
of interests.

\subsubsection*{Names}

Names are available in all languages (name, name\_en, name\_de,
name\_fr, name\_es, name\_ru, name\_zh).

\subsubsection*{Scalerank}

The scalerank of a POI is determined by the area.

\begin{table}[H]
\label{my-label}
\begin{tabular}{l | l}
Scalerank & Description                                       \\ \hline
1         & The POI has a very large area \textgreater145000  \\
2         & The POI has a medium-large area \textgreater12780 \\
3         & The POI has a small area 2960 or is a station     \\
4         & The POI has no known area                        
\end{tabular}
\end{table}

\subsubsection*{Localrank}

The localrank field can be used to adjust the label density by showing
fewer labels. The localrank is a whole number which starts at 1 and
groups places in a grid by their importance.

Importance of POIs are weighted in the following order:

\begin{enumerate}
\def\labelenumi{\arabic{enumi}.}
\item
  station, subway\_entrance, park, cemetery, bank, supermarket, car,
  library, university, college, police, townhall, courthouse
\item
  nature\_reserve, garden, public\_building
\item
  stadium
\item
  hospital
\item
  zoo
\item
  university, school, college, kindergarten
\item
  supermarket, department\_store
\item
  nature\_reserve, swimming\_area
\item
  attraction
\end{enumerate}

\subsubsection*{Maki Labels and Types}\label{maki-labels-and-types}

The \texttt{type} field in the poi\_label layer is mapped to the appropriate Maki
label which can be queried in maki. Types are stored in a human readable
format in the data where chair\_lift becomes Chair lift so you can use
the type field for as label.
