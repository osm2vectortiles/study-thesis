\chapter{Classification}
\label{classification}

The OpenStreetMap tagging schema has developed into a complex taxonomy of real-world feature classes and objects. \cite[p. 15]{haklay2008openstreetmap}. Map designers don't want to design
for each distinct object specifically which is why Mapbox and others abstract distinct key value pairs into so called feature classes.

Mapbox calls those feature class simply class.

Mapping created key value pairs into categories cannot be automated
and there is no standard. This is why we have done it by hand.

A map designer that wants to style agricultural areas does not care
what type of field it is.

\begin{table}[]
\centering
\caption{My caption}
\label{my-label}
\begin{tabular}{llll}
Key      & Value      & Class       & Type           \\
landuse  & farm       & agriculture & orchard        \\
building & farm       & agriculture & farm           \\
landuse  & farmland   & agriculture & farmland       \\
landuse  & farmyard   & agriculture & farmyard       \\
landuse  & allotments & agriculture & allotments     \\
landuse  & vineyard   & agriculture & vineyard       \\
landuse  & vineyard   & agriculture & plant\_nursery
\end{tabular}
\end{table}

\section{Classification Format}

We created the classifications in a YAML based format.
Where each key in \texttt{classifications} denotes the classification name. The elements within each classification (e.g. \texttt{driveway} or \texttt{main} are the class name and the values below the class name (e.g. \texttt{primary}, \texttt{primary\_link} the OSM values to match. We do not explicitly match the OSM keys as well - only the values.

\begin{yamlcode}
classifications:
  road:
    highway:
    - motorway
    - motorway_link
    - driveway
    main:
    - primary
    - primary_link
    - trunk
    - trunk_link
    - secondary
    - secondary_link
    - tertiary
    - tertiary_link
\end{yamlcode}

\section{Code Generation}

We take the readable classification format and generate immutable SQL
functions we can use in our queries.
The example above will result in the following function.

\begin{sqlcode}
CREATE OR REPLACE FUNCTION classify_road(type VARCHAR)
RETURNS VARCHAR AS $$
  BEGIN
    RETURN CASE
      WHEN type IN ('motorway','motorway_link','driveway') THEN 'highway'
      WHEN type IN ('primary','primary_link',
                    'trunk','trunk_link',
                    'secondary','secondary_link',
                    'tertiary','tertiary_link') THEN 'main'
    END;
  END;
$$ LANGUAGE plpgsql IMMUTABLE;
\end{sqlcode}

\section{Use in Vector Tile Generation}

Classifications are then baked into vector tile attributes
of geometries.

\begin{sqlcode}
SELECT
  geometry,
  classify_road(type) AS class,
  type AS type
FROM osm_roads
\end{sqlcode}