\chapter{Results and Future}\label{part2_results_and_future}

\section{Results}\label{part2_results}
The result of this study thesis are described in Part 1, \hyperref[part1_results]{\emph{section 4.1}}.

\section{Future Development}\label{part2_future_development}
The first version of the rendered vector tiles has shown, that it is possible to get very close to compatibility with Mapbox Streets in a reasonable amount of time. There are still some quirks, which need to be ironed out, but we created a good foundation for future development. The two sections below describe small improvements and new features, which will be implemented in the bachelor thesis.

\subsection{Small improvements}\label{small_improvements}
 
\paragraph{Labels}
The scalerank of the place, marine, state and road labels should be improved. Th current implementation is good for the moment, but still not equal to Mapbox Streets.

\paragraph{State Labels}
State labels are used to display states of big countries like USA, Russia or China. More state labels could be added. 

\paragraph{Country boundaries}

A mix of Natural Earth and OSM data is used for the administrative boundaries.
This layer is using both data sources, because OSM has some issues with marine borders and simplified borders on lower zoom levels.
Using Natural Earth data alone is not suitable, because the data is generalized. Therefore it does not look good on higher zoom levels. For the next release a better data source should be chosen for administrative boundaries.

\paragraph{Country Name Translation Rule}
Mapbox has defined a fallback rule\cite{22_mapbox.com_2015} for the country names. Due to time limits this behavior could not be implemented and should be implemented in a future release.

\paragraph{Special Maki Icons for US Road Labels}
Mapbox uses different maki icons\cite{101_mapbox.com_2015} for the road labels in the US.  

\paragraph{Possiblity of Mapbox Terrain Visual Style}
Check if all the data required for a Mapbox style like Mapbox Terrain\cite{102_mapbox.com_2015} is included in the vector tiles

\paragraph{Exclude Water in Rendering Process}

When the vector tile rendering process was implemented, it was realized that a lot of rendering time could be saved
when the water (no data) is excluded.
It turned out, that deciding if a vector tiles will have no data, is not an easy problem. At the current stage of the project
everything is rendered and the "empty" vector tiles are removed afterwards.

\subsection{New Features}\label{new_features}

\paragraph{Vector Tiles of Entire World}
One of the main targets of the bachelor thesis is to render the vector tiles of the entire world. This brings new challenges like scaling the rendering infrastructure. 

\paragraph{Update Vector Tiles}
A big request of the Swiss OSM community was to provide updated vector tiles based on the diff\cite{5_wiki.openstreetmap.org_2015} files.
This also requires detecting dirty tiles in the already rendered vector tiles because one update can affect multiple tiles across multiple zoom levels.

\paragraph{Faster Tile Server}
There is no robust and scalable raster tile server based on vector tiles available yet.
This may even be a separate project.

\paragraph{Geographic Name Search}

To fulfill the long term goal of this project to provide an offline map a basic geographic name search could be implemented.

\newpage{}