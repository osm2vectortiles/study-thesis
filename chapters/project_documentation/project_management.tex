\chapter{Project Management}\label{project-management}

\section{GitHub}\label{github}

\section{Prototypen, Releases, Meilensteine}

\section{Roles and Responsibilities}\label{roles-and-responsibilities}

\section{Aufwandschätzung, Zeitplan, Projektplan}

\section{Risks}\label{risks}

\section{Prozessmodell}




Managing tickets, milestones and progress all happens on the public
GitHub repository.

There are two GitHub repositories:

\begin{itemize}
\tightlist
\item
  \textbf{osm2vectortile} contains the project
\item
  \textbf{osm2vectortile-thesis} contains the thesis
\end{itemize}

\subsection{Development Environment}\label{development-environment}

Development happens independant.

\subsection{Version Control}\label{version-control}

Git

\subsection{Continuous Integration}\label{continuous-integration}

TBD

\section{Quality Measures}\label{quality-measures}

\subsection{Documentation}\label{documentation}

\subsection{Project Management}\label{project-management-2}

\section{Guidelines}\label{guidelines}

\subsection{Releases}\label{releases}

We use semantic versioning \footnote{\url{http://semver.org/}}. At the
end of each iteration a new release will be created. We start at version
\texttt{0.1.0}. The version number consists of a major, minor and patch
version.

\textbf{Major: }Incompatible changes

\textbf{Minor: }Backwards compatible changes

\textbf{Patch: }Bugfixes

\subsection{Git}\label{git}

\subsubsection{Commit Messages}\label{commit-messages}

We use the seven rules of great git commit
messages\footnote{\url{http://chris.beams.io/posts/git-commit/}}.

\begin{itemize}
\item
  Separate subject from body with a blank line
\item
  Limit the subject line to 50 characters
\item
  Capitalize the subject line
\item
  Do not end the subject line with a period
\item
  Use the imperative mood in the subject line
\item
  Wrap the body at 72 characters
\item
  Use the body to explain what and why vs.~how
\end{itemize}

\subsubsection{History}\label{history}

\paragraph{Rewriting}\label{rewriting}

Git history should be kept clean and therefore local branches should be
squashed meaningfully.

\paragraph{Pulling}\label{pulling}

To avoid unnecessary merge messages one should always use the
\texttt{-\/-rebase} parameter.

\subsection{Workflow}\label{workflow}

We use the Feature Branch
Workflow\footnote{\url{https://www.atlassian.com/git/tutorials/comparing-workflows/feature-branch-workflow/}}.

Every project member has a local repository with a copy of the remote
repository (GitHub). For each feature ticket in GitHub a separate branch
will be created. Once a ticket has been completed a pull request will be
created and needs to be merged into the Master branch by an other team
member.

Every team member is responsible that the code is reviewed by the other
team mebmer. The reviewer will add his comments to the pull requst
comment section. As soon as all issues have been resolved and continuous
integrations runs successfully the branch will be merged into master.

\subsection{Coding Standards}\label{coding-standards}

\subsection{Testing}\label{testing}

\subsubsection{Unit Tests}\label{unit-tests}

At the moment there is no need for unit tests.

\subsubsection{Integration Tests}\label{integration-tests}

To guarantee our solution works we will create a few completly
integrated integration tests that cover our core use cases.

\section{Planning}\label{planning}

\subsection{Work Schedule}\label{work-schedule}

The planning process defies the milestones and bigger tasks that need to
be resolved.