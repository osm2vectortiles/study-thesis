\chapter{Design}\label{design}


%------------------------------------------------------
\section{Architecture}\label{architecture}



%------------------------------------------------------
\section{Domain Model}



%------------------------------------------------------
\section{Package Diagrams}


%------------------------------------------------------
\section{Workflow}\label{workflow}

\paragraph{First step: Getting OSM
data}\label{first-step-getting-osm-data}

There are many different sources available to get OSM data from. Most of
the time Geofabrik\footnote{\url{http://download.geofabrik.de/}} was
referenced for getting single countries, continents or even the whole
planet. But many times people only need a single city or region, because
of this demand
Mapzen\footnote{\url{https://mapzen.com/data/metro-extracts/}} provides
OSM data for many cities and regions around the globe.

The OSM data is missing something very important: the administrative
boundaries. This needs to be downloaded separatly due to the fact, that
somebody could manipulate the boundary of a region. As a result of this
administrative boundaries get checked by the OSM community and released
separatly.

The data is available in the PBF and OSM XML format. If available the
PBF(Protocolbuffer Binary
Format)\footnote{\url{http://wiki.openstreetmap.org/wiki/PBF_Format}}
version should be choosen, as it is 30\% smaller and 5-6 times faster to
read and write than the bzipped OSM XML version.

\paragraph{Second step: Importing OSM data into
Postgis}\label{second-step-importing-osm-data-into-postgis}

As with the data sources there are many possiblities to import and store
OSM data. The database type and importing schema should be choosen based
on the primary use case. Since the goal of this workflow is to render
vector tiles, our primary use case is rendering. The OSM community
recommends\footnote{\url{http://wiki.openstreetmap.org/wiki/Databases_and_data_access_APIs}}
PostgreSQL with the Postgis extension and imposm or osm2pgsql as schema.

\subparagraph{imposm importer}\label{imposm-importer}

Imposm is an import tool for osm data, it is not a schema. But it
defines a default
schema\footnote{\url{http://imposm.org/docs/imposm/latest/database_schema.html}},
which could possibly be changed by provinding a custom mapping file. An
advantage of the default schema is that it groups data thematically into
tables. Which results in smaller tables and simpler queries. Imposm 3
supports updating the database from OSM diff
files\footnote{\url{http://imposm.org/docs/imposm3/latest/tutorial.html\#diff}}

\subparagraph{osm2pgsql importer}\label{osm2pgsql-importer}

Osm2pgsql is must commonly used to import OSM data for the rendering use
case. The import schema is also called osm2pgsql and defines a very
simple schema(line, point, polygon and
roads)\footnote{\url{http://wiki.openstreetmap.org/wiki/Osm2pgsql/schema}}.
This results in very large tables, so it is recommended to create good
indices. Osm2pgsql supports updating of the database, if the values have
been stored as hstore.

For our use case it is important, that the import is efficent and that
the import tool supports updating based on OSM diff files. Imposm 3 is
faster than osm2pgsql and supports updatability. So we decided to take
imposm for importing.

\subsubsection{Third step: Mapbox studio source
project}\label{third-step-mapbox-studio-source-project}

A Mapbox studio source project is divided into the following folder
structure\footnote{\url{https://www.mapbox.com/guides/source-manual/\#source-project}}:

\begin{verbatim}
source-project.tm2source/
    data.yml
    data.xml
    .thumb.png
\end{verbatim}

The data file defines all feature sets(layers) like landuse, waterway,
road etc. The definition contains metadata like id, datasource(db, host,
query, srid, extent), description, fields and properties. Mapbox Studio
needs the yml version and mapnik the xml version of this file.
.thumb.png is a thumbnail image that gets displayed in the projects
list.

\subsubsection{Fourth step: Generating vector
tiles}\label{fourth-step-generating-vector-tiles}

To generate the vector tiles we use mapnik. Mapbox provides a very handy
tool to generate vector tiles.

\begin{verbatim}
tilelive-copy --bounds=-180,-85,180,85 bridge:///home/mapbox/tmp/project.tm2source/data.xml mbtiles:///tmp/project.mbtiles
\end{verbatim}

tilelive-copy provides the Mapnik XML file and the extent to mapnik
which then generates the vector tiles. Tilelive-copy outputs these
vector tiles in the mbtiles container.

%------------------------------------------------------